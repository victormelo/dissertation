%!TEX root = ../dissertation_vkslm.tex

\chapter{Introduction}
% ----------------------------------------------------------

In the modern society, biometric technology is used in several security applications for personal authentication. The aim of such systems is to confirm the identity of a given subject based on physiological or behavioral traits. In the first case, recognition is based on biological characteristics such as fingerprint, palm print, iris, face. The latter relies on behavioral traits such as voice pattern and handwritten signature \cite{jain2004biometrics}.

The Handwritten Signature is, so far, one of the primary methods for identity authentication: signature acquisition is easy, non-invasive, and most individuals are familiar with its use in their daily life \cite{impedovo2008state}. Due to its convenient nature, signatures can be employed as a sign of confirmation in a wide variety of documents, namely, bank checks, identification documents and a variety of business certificates and contracts.

As a behavioral trait, signatures present a high intra-user variability and are susceptible to spoof attacks, which is the attempt to forge the signature of a legitimate subject \cite{jain2004biometrics}. Two types of impostors are considered: casual impostors (producing random forgeries) when no information about authentic writer signature is known, and real impostors (producing skilled forgeries) when some information of the signature is used \cite{fierrez2008handbook}.


If a signature on a document is forged, this document is also considered invalid. Thus, preventing frauds in the signature verification process has been a challenge for researchers around the world. However, manual signature-based authentication is a time-consuming, heavy-load and expansive task. Hence, several Automatic Handwritten Signature Verification Systems (AHSVS) have been proposed \cite{impedovo2008state}. These systems aim to automatically decide if a query signature is genuine or not.

An AHSVS is essentially a pattern recognition system that receives a signature as input, extracts a set of features and classifies the sample using a template as the reference. Two acquisition modalities are available based on the specific application: signatures can be acquired by using an optical scanner or by using an acquisition device such as digitizing tablets or electronic pens with digital ink. The two domains are identified as offline (static) \cite{hafemann2015offline} and online (dynamic) \cite{cpalka2014line}, respectively. In the offline case, only a static representation of the completed writing is available as an image, while in the online modality data is stored during the writing process and consists of a temporal sequence of values. Each representation has, in general, specific attributes not present in the other \cite{viard1999ireste}.

An AHSVS is essentially a pattern recognition system that receives a signature as input, extracts a feature set from the data and classifies the sample using a template database as the reference. These datasets contain signatures digitized either by using an optical scanner to obtain the signature directly from the paper or using an acquisition device such as digitizing tablets or electronic pens with digital ink. The two domains are identified as offline (static) and online (dynamic), respectively.  In the online modality, data is stored during the writing process and consists of a temporal sequence of the two-dimensional coordinates (x, y) of consecutive points, whereas in the offline case, only a static representation of the completed writing is available as an image. Moreover, each representation has specific attributes not present in the other \cite{viard1999ireste}. 

Performance of an AHSVS also depends upon the amount of samples used for training. In order to increase the amount of training samples, synthetic signatures have been used for the online domain  \cite{galbally2009synthetic, galbally2012synthetic} as well as for the offline one \cite{ferrer2013synthetic, ferrer2013realistic}.

Static signature images can be generated from online data  \cite{diaz2014generation, guest2013assessment, ferrer2013realistic, galbally2015line}. This approach can be used also to improve performance of an online signature verification system by coupling it with a parallel offline one \cite{chapter}.

However it must be argued that despite the advancements for this category of synthesis, the synthetic offline samples created up to date still struggle to improve the recognition rates when used to enlarge existing offline signature databases.

In this work, an approach based on Fully Convolutional Neural Networks (FCN) trained in a supervised manner is presented with the aim to learn the representation of offline manuscripts using as the input the online representation, expecting that the network predicts the corresponding static sample. This type of approach is interesting since the neural network learns how the dynamic features, in particular pressure, of the online representation translates into the static domain (pen on the paper).

Differently from other models \cite{ferrer2013synthetic, ferrer2013realistic, diaz2014generation}, the approach here presented is designed under the perspective of supervised training: a learning model is trained to perform the task of “online to offline conversion” using (for training purpose) a data set containing both the online and offline versions of manuscripts. We expect that the Deep Neural Network model can learn how online information is transformed into the offline manuscript so that synthetic signatures can be used to improve recognition rates). 

In order to achieve this goal, the following steps have been performed:
\begin{inlinelist}
	\item creation and training of a Deep Neural Network model able to translate dynamic handwritten information into an offline manuscript
	\item generation of an offline synthetic dataset based on a publicly available online signature dataset
	\item comparison of the performance with a state-of-the-art classifier to evaluate the closeness of synthetic and real signatures
\end{inlinelist}. We also evaluate the use of synthetic signatures in the genuine enrollment set, analyzing the differences of the offline verification system recognition rates.

\section{Problem Statement}
Encouraged by the motivations depicted previously the goal of this dissertation can be stated as follows:

\textit{"The goal of this work is to design an approach to generate synthetic offline handwriting signatures based on online data, modeling this problem as a supervised machine learning task, through a Deep Convolutional Neural Network, in order to enlarge offline signature datasets to improve offline signature verification systems recognition rates."}

This statement is developed through the following actions:
\begin{inlinelist}
\item Creation and training of a Deep Neural Network model able to translate dynamic handwritten information into an offline manuscript
\item Generation of an offline synthetic dataset based on a publicly available online signature dataset
\item Comparison of the proposed approach’s performance with a state-of-the-art method to evaluate the closeness of synthetic signatures with respect to real signatures. 
\end{inlinelist}

To achieve point (iii), machine-oriented validations are carried out. A state-of-the-art automatic signature verifier is used to evaluate our synthetic samples versus real offline signatures from a publicly available dataset. It is expected that similar results would be obtained with both the real and our proposed method synthetic signatures. Moreover, we also evaluate the use of synthetic signatures in the genuine enrollment set, analyzing the differences in the offline verification system recognition rates. 

\section{Dissertation Structure}
From this introduction, the remainder of this work is organized as follows:

Chapter \ref{ch:sig} gives an overview on the topic of handwritten signature verification. We introduce handwritten signatures as a biometric trait, characterize how an automatic handwritten signature verification system works and summarize previous studies performed in the context of synthetic offline signatures from online data. In Chapter \ref{ch:nndl}, we give a context and describe the  Deep Learning research field. Chapter \ref{ch:method} describes our proposed method. Chapter \ref{ch:exp} presents the experimental protocol followed in the experiments of this work and Chapter \ref{ch:results} presents and discuss the results. Finally, in Chapter \ref{ch:conclusion} we present the conclusion of this study together with suggestions for future work.
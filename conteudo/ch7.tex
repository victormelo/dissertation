%!TEX root = ../dissertation_vkslm.tex

\chapter{Conclusion and Future Works}\label{ch:conclusion}
This work proposes a method to synthesize off-line signatures from dynamic information. Our Fully Convolutional Neural Network model learns end-to-end how to map online manuscripts into static samples taking advantage of the data present in the IRONOFF dataset during the training phase of the Neural Network. 

We show that, in contrast to what already have been proposed, it is possible to model the "online to offline signature" conversion as a supervised learning task. Our Deep Neural Network model learns how the online biometric data must be used to build the grayscale signature image.
   
We observe that our proposed synthetic images  synthesizes offline signatures offer a verification performance similar to the one offered by real signatures and we show that the synthetic signatures present high discriminative power when used to increase the enrollment set under the skilled forgeries scenario, one of the biggest challenges in signature verification. 

Future work can explore the optimization of the hyperparameters of the convolutional neural
network (such as number of layers, number of neurons per layer, etc.), since these hyper-
paramenters were fixed during the experiments on the present work.

Besides, we intend to explore the combination of real online and synthetically generated offline signatures the proposed method, when only the online information is available, towards improved recognition results.

Another improvement to the current work is the integration of other dynamic features in addition to the pressure on the input of the network, such as the velocity. Moreover, one could also design a model to synthesize bigger resolution offline signatures.
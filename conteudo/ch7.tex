%!TEX root = ../dissertation_vkslm.tex

\chapter{Conclusion and Future Works}\label{ch:conclusion}

In this work a method to produce synthetic offline signatures by means of a supervised learning task has been presented. We propose a Fully Convolutional Neural Network to learn an end-to-end mapping of online manuscripts to the static domain, taking advantage of the data present in the IRONOFF dataset during the training phase of the Neural Network. 

We show that, in contrast to what has already been proposed, it is possible to model the "online to offline signature" as a learning from data task. Our Deep Neural Network model learns how the online biometric data must be used to build the grayscale signature image.  The use of synthetically generated signatures offers verification performance similar to those observed by using real signatures. We also show that the synthetic signatures generated with our proposed method present high discriminative power when used to increase the enrollment set under the skilled forgeries scenario. 

In order to evaluate our proposed method, we evaluated our proposed method by means of machine-oriented validations. We used a state-of-the-art automatic signature verifier to assess that similar results were obtained with both the real and our proposed method synthetic signatures. We found that our proposed method produces synthetic signatures comparable to real ones and when used to increase amount of training samples the system's EER decreases from 10.26\% to 9.74\% on the random forgeries scenario and from 21.55\% to 19.17\% for skilled forgeries.

Future work can explore the optimization of the hyper-parameters of the FCN (such as the number of layers, number of neurons per layer) since these hyper-parameters were fixed during the experiments in the present work. Besides, we intend to explore the combination of real online and synthetically generated offline signatures using the proposed method, when only the online information is available, towards improved recognition results on a dynamic signature verifier.

It is also possible to explore recent advancements on Generative Adversarial Networks \cite{goodfellow2014generative} such as the Image-to-image translation \cite{isola2016image} method to convert the online samples to static images.

Another improvement to the current work is the integration of other dynamic features in addition to the pressure on the input of the network, such as the velocity. Moreover, one could also design a model to synthesize offline signatures with bigger resolution.
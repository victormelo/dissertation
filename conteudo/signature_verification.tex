%!TEX root = ../dissertation_vkslm.tex

\chapter{Synthetic handwritten signatures}
\section{Automatic Handwritten signature verification systems}
In this context, the scientific
literature has proposed two ways to synthesize the signa-
tures [7], [8]:

1) Generation of duplicated samples. In this case algo-
rithms create artificial intra-personal variability from reference
specimens. These reference specimens are the seed of these
algorithms. Therefore, they do not create new individuals but
possible signatures executed by the signers. Both in on-line [9],
[10], [11], [12] and off-line [13], [14], [15], [16], [17], [18],
[19], [20], the algorithms are able to generate new samples of
a real user. Duplicated signatures present several advantages
such as training improvement in ASV, statistically meaningful
evaluations, enlarging the number of samples in a database,
etc.

2) Generation of new identities. Algorithms are designed
in this modality by taking into account different rules and
logics in order to simulate realistic effects. In [21] a model
is described to generate dynamic signatures based on visual
characteristics extracted from the time domain. The method
was evaluated from a visual validation. In Galbally et al. [7],
[8] the full generation of on-line flourish-based signatures was
carried out through two algorithms based on spectral analysis
and the kinematic theory of rapid human movements. The
first attempt to generate static flourishes in context of static
signature image is due to [22]. Two follow-up works were
presented in [1], [2] where authors coped with the generation
of Western static and dynamic signatures (text and flourish) as
well as skilled forgeries under a neuromotor approach.

\section{Generation of duplicated samples}

\section{Generation of new identities}

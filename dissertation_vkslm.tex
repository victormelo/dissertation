%% abtex2-modelo-trabalho-academico.tex, v-1.9.6 laurocesar
%% Copyright 2012-2016 by abnTeX2 group at http://www.abntex.net.br/
%%
%% This work may be distributed and/or modified under the
%% conditions of the LaTeX Project Public License, either version 1.3
%% of this license or (at your option) any later version.
%% The latest version of this license is in
%%   http://www.latex-project.org/lppl.txt
%% and version 1.3 or later is part of all distributions of LaTeX
%% version 2005/12/01 or later.
%%
%% This work has the LPPL maintenance status `maintained'.
%%
%% The Current Maintainer of this work is the abnTeX2 team, led
%% by Lauro César Araujo. Further information are available on
%% http://www.abntex.net.br/
%%
%% This work consists of the files abntex2-modelo-trabalho-academico.tex,
%% abntex2-modelo-include-comandos and abntex2-modelo-references.bib
%%

% ------------------------------------------------------------------------
% ------------------------------------------------------------------------
% abnTeX2: Modelo de Trabalho Academico (tese de doutorado, dissertacao de
% mestrado e trabalhos monograficos em geral) em conformidade com
% ABNT NBR 14724:2011: Informacao e documentacao - Trabalhos academicos -
% Apresentacao
% ------------------------------------------------------------------------
% ------------------------------------------------------------------------

\documentclass[
	% -- opções da classe memoir --
	12pt,				% tamanho da fonte
	openright,			% capítulos começam em pág ímpar (insere página vazia caso preciso)
	oneside,			% para impressão em recto e verso. Oposto a oneside
	a4paper,			% tamanho do papel.
	sumario=tradicional, % modelo de sumário
	% -- opções da classe abntex2 --
	%chapter=TITLE,		% títulos de capítulos convertidos em letras maiúsculas
	%section=TITLE,		% títulos de seções convertidos em letras maiúsculas
	%subsection=TITLE,	% títulos de subseções convertidos em letras maiúsculas
	%subsubsection=TITLE,% títulos de subsubseções convertidos em letras maiúsculas
	% -- opções do pacote babel --
	brazil,			% idioma adicional para hifenização
	french,				% idioma adicional para hifenização
	spanish,			% idioma adicional para hifenização
	english				% o último idioma é o principal do documento
	]{ppgec-abntex2}

% ---
% Pacotes básicos
% ---
\usepackage{times}			    % Usa a fonte Times New Roman
\usepackage[T1]{fontenc}		% Selecao de codigos de fonte.
\usepackage[utf8]{inputenc}		% Codificacao do documento (conversão automática dos acentos)
\usepackage{lastpage}			% Usado pela Ficha catalográfica
\usepackage{indentfirst}		% Indenta o primeiro parágrafo de cada seção.
\usepackage{color}				% Controle das cores
\usepackage{graphicx}			% Inclusão de gráficos
\usepackage{microtype} 			% para melhorias de justificação
% ---
% used for smart inline list
\usepackage{enumitem}
\setlist[enumerate,1]{%
  label=\arabic*.,
}

\newlist{inlinelist}{enumerate*}{1}
\setlist*[inlinelist,1]{%
  label=(\roman*),
}
% ---
% Pacotes adicionais, usados apenas no âmbito do Modelo Canônico do abnteX2
% ---
\usepackage{lipsum}				% para geração de dummy text
% ---

% ---
% Pacotes de citações
% ---
\usepackage[num]{abntex2cite}	% Citações padrão ABNT
\citebrackets[]
% hack para colocar colchetes nas referencias.
\makeatletter
\ifthenelse{\boolean{ABCIbiblabelonmargin}}
{
\renewcommand{\@biblabel}[1]%
{\ifthenelse{\equal{#1}{}}{}{{\citenumstyle #1\hspace{\biblabelsep}}}}
}
{
\renewcommand{\@biblabel}[1]%
{%
\ifthenelse{\equal{#1}{}}
{}
{%
\def\biblabeltext{{\citenumstyle [#1]\hspace{\biblabelsep}}}%
\settowidth{\ABCIauxlen}{\biblabeltext}%
\ifthenelse{\lengthtest{\ABCIauxlen<\minimumbiblabelwidth}}
{\setlength{\ABCIauxlen}{\minimumbiblabelwidth-\ABCIauxlen}}
{\setlength{\ABCIauxlen}{0cm}}%
{\biblabeltext\hspace{\ABCIauxlen}}%
}%
}%
}
\makeatother
% ---
% Informações de dados para CAPA e FOLHA DE ROSTO
% ---
% ---
% Informações de dados para CAPA e FOLHA DE ROSTO
% ---

\titulo{A Deep Learning Approach to Generate Offline Handwritten Signatures Based on Online Samples}
\autor{Victor Kléber Santos Leite Melo}
\local{Recife}
\data{august, 2017}
\orientador{Prof. Dr. Byron Leite Dantas Bezerra}
\coorientador{Prof. Dr. Giuseppe Pirlo}
\instituicao{
  \SingleSpacing
  Universidade de Pernambuco \\
  Escola Politécnica de Pernambuco \\
  Programa de Pós-Graduação Acadêmica em Engenharia de Computação
}
\tipotrabalho{Master's Degree Dissertation}
% O preambulo deve conter o tipo do trabalho, o objetivo,
% o nome da instituição e a área de concentração
\preambulo{Dissertation presented to the Programa de Pós-Graduação acadêmico em ENGENHARIA DE COMPUTAÇÃO da Universidade de Pernambuco as a partial requisite to
obtain the degree of Master of Science in Computer Engineering.}
% ---



% ---
% Configurações de aparência do PDF final

% alterando o aspecto da cor azul
\definecolor{blue}{RGB}{41,5,195}

% informações do PDF
\makeatletter
\hypersetup{
     	%pagebackref=true,
		pdftitle={\@title},
		pdfauthor={\@author},
    	pdfsubject={\imprimirpreambulo},
	    pdfcreator={LaTeX with abnTeX2},
		pdfkeywords={abnt}{latex}{abntex}{abntex2}{trabalho acadêmico},
		colorlinks=false,       		% false: boxed links; true: colored links
    	linkcolor=blue,          	% color of internal links
    	citecolor=blue,        		% color of links to bibliography
    	filecolor=magenta,      		% color of file links
		urlcolor=blue,
		bookmarksdepth=4
}
\makeatother
% ---

% ---
% Espaçamentos entre linhas e parágrafos
% ---

% O tamanho do parágrafo é dado por:
\setlength{\parindent}{1.3cm}

% Controle do espaçamento entre um parágrafo e outro:
\setlength{\parskip}{0.2cm}  % tente também \onelineskip

% ---
% compila o indice
% ---
\makeindex
% ---

%%% -----
%%% Formato de cabeçalho/rodapé romano nos elementos pré-textuais
%%% -----

%% Novo estilo
\makepagestyle{estilo_pretextual} %%% escolha um nome
\makeevenhead{estilo_pretextual}{}{}{\ABNTEXfontereduzida \textit \thepage}
\makeoddhead{estilo_pretextual}{}{}{\ABNTEXfontereduzida \textit \thepage}

%% Customiza comando \pretextual
\renewcommand{\pretextual}{

  \aliaspagestyle{chapter}{estilo_pretextual}% customizing chapter pagestyle
  \pagestyle{estilo_pretextual}
  \aliaspagestyle{cleared}{empty}
  \aliaspagestyle{part}{estilo_pretextual}
}

% ---
% Ajusta a marca \textual para que a numeração volte a ser arábica
% nos elementos textuais
\let\oldtextual\textual        % copia o comando \textual anterior para \oldtextual
\renewcommand{\textual}{%
  \oldtextual%
  \pagenumbering{arabic} % volta à numeração arábica
}
% ---



% ----
% Início do documento
% ----
\begin{document}

% Seleciona o idioma do documento (conforme pacotes do babel)
\selectlanguage{english}
%\selectlanguage{brazil}

% Retira espaço extra obsoleto entre as frases.
\frenchspacing

% ----------------------------------------------------------
% ELEMENTOS PRÉ-TEXTUAIS
% ----------------------------------------------------------


% ---
% Capa
% ---
\imprimircapa
% ---

% ---
% Folha de rosto
% (o * indica que haverá a ficha bibliográfica)
% ---
\thispagestyle{empty}
\imprimirfolhaderosto*
% ---

% Inicia a numeração dos elementos pre textuais após a folha de rosto
\pagenumbering{roman} %%% ou \pagenumbering{Roman}

% ---
% Inserir a ficha bibliografica
% ---

% Porém, a biblioteca da sua lhe fornecerá um PDF
% com a ficha catalográfica definitiva após a defesa do trabalho. Quando estiver
% com o documento, salve-o como PDF no diretório do seu projeto e substitua todo
% o conteúdo de implementação deste arquivo pelo comando abaixo:
%
% \begin{fichacatalografica}
%     \includepdf{fig_ficha_catalografica.pdf}
% \end{fichacatalografica}

% ---

% ---
% Inserir folha de aprovação
% ---

% Isto é um exemplo de Folha de aprovação, elemento obrigatório da NBR
% 14724/2011 (seção 4.2.1.3). Você pode utilizar este modelo até a aprovação
% do trabalho. Após isso, substitua todo o conteúdo deste arquivo por uma
% imagem da página assinada pela banca com o comando abaixo:
%
% \includepdf{folhadeaprovacao_final.pdf}
%

% ---

% ---
% Dedicatória
% ---

\begin{dedicatoria}
   \vspace*{\fill}
   \centering
   \noindent
   \textit{ à minha família. } \vspace*{\fill}
\end{dedicatoria}

% ---

% ---
% Epígrafe
% ---
\begin{epigrafe}
    \vspace*{\fill}
	\begin{flushright}
O epígrafe é uma pagina onde o autor pode colocar uma citação ou pensamento que, de alguma forma, influenciou em seu trabalho acadêmico.
	\end{flushright}
\end{epigrafe}
% ---

% ---
% RESUMOS
% ---

% resumo em inglês
\setlength{\absparsep}{18pt} % ajusta o espaçamento dos parágrafos do resumo
\begin{resumo}
One of the main challenges of off-line signature verification is the absence of large databases. A possible alternative to overcome this problem is the generation of synthetic signature databases. In this work, a novel method for the generation of synthetic off-line signatures based on dynamic information is presented. In contrast to the state-of-the-art, we propose a synthesis approach under the perspective of supervised training, in which our learning model is trained to perform the task of ``online signature to off-line signature conversion''. The proposed approach is based on a Deep Convolutional Neural Network trained to learn how online manuscripts of the IRONOFF dataset are transformed into the offline domain. The main goal of the proposed method is to synthetically enlarge existing off-line signature datasets based on online signature samples towards an improvement on the recognition rates of off-line signature verification systems. For these purposes, a machine-oriented evaluation on the BiosecurID signature dataset is carried out. We show that our Deep Neural Network model can learn how the online biometric data must be used to build the grayscale signature image. Specifically, we observe that the synthetic samples generated with our proposed method are comparable to real signatures, achieving a verification performance similar to the one offered by real signatures. We also show that the proposed system can be combined with any other offline signature verification system to increase the number of samples on the enrollment set, leading to improved EER, including when used to detect skilled forgeries.
\end{resumo}

% resumo em português
\begin{resumo}[Resumo]
 \begin{otherlanguage*}{brazil}
  Um dos principais desafios de sistemas de verificação de assinaturas \textit{offline} é a ausência de grandes conjunto de dados. Uma alternativa possível para superar esse probema é a geração de assinaturas sintéticas. Neste trabalho é proposto um método para a geração sintética de assinaturas offline baseado em características dinâmicas. Em contraste com o estado-da-arte, o método  proposto se baseia na perspectiva da aprendizagem supervisionada, o nossa máquina de aprendizagem é treinada para realizar a tarefa de ``conversão de assinatura \textit{online} para \textit{offline}''. O método proposto é uma \textit{Deep Convolutional Neural Network} treinada para aprender como textos manuscritos \textit{online} da base IRONOFF são transformados para o domínio \textit{offline}. O objetivo principal do método proposto é o de aumentar sinteticamente bases de assinatura \textit{offline} baseando-se em amostras \textit{online} em direção a uma melhora nas taxas de reconhecimento de sistemas de verificação de assinaturas \textit{offline}. Para isso, uma avaliação na base de assinaturas BiosecurID é realizada. Mostra-se que o modelo de rede profunda proposto consegue aprender como converter os dados biométricos \textit{online} para criar a respectiva imagem em escala de cinza da assinatura. Especificamente, observa-se que as amostras sintéticas geradas com o método proposto são comparáveis a amostras reais, obtendo uma performance comparável ao de assinaturas reais. Também é mostrado que o sistema proposto pode ser combinado com qualquer sistema de verificação de assinatura \textit{offline} para aumentar o número de amostras no conjunto de treinamento, levando a uma melhora no \textit{EER}, inclusive quando usado para detectar forjas habilidosas.
  
   \vspace{\onelineskip}
 
   \noindent 
 \end{otherlanguage*}
\end{resumo}


% ---



% ---
% inserir o sumario
% ---
\pdfbookmark[0]{\contentsname}{toc}
\tableofcontents*
\cleardoublepage
% ---


% ---
% inserir lista de ilustrações
% ---
\pdfbookmark[0]{\listfigurename}{lof}
\listoffigures
\cleardoublepage
% ---

% ---
% inserir lista de tabelas
% ---
\pdfbookmark[0]{\listtablename}{lot}
\listoftables
\cleardoublepage
% ---

% ---
% inserir lista de siglas e simbolos
% ---

\begin{siglas}
	\addcontentsline{toc}{chapter}{\listadesiglasname}
	
  \item[ABNT] Associação Brasileira de Normas Técnicas
  \item[abnTeX] ABsurdas Normas para TeX
  \item[$ \Gamma $] Letra grega Gama
  \item[$ \Lambda $] Lambda
  \item[$ \zeta $] Letra grega minúscula zeta
  \item[$ \in $] Pertence
\end{siglas}


% ---



% ---
% Agradecimentos
% ---
\begin{agradecimentos}

\lipsum[1-1]


\end{agradecimentos}
% ---


% ----------------------------------------------------------
% ELEMENTOS TEXTUAIS
% ----------------------------------------------------------
\textual
%\pagestyle{simple}
%\aliaspagestyle{chapter}{simple}

%!TEX root = ../dissertation_vkslm.tex

\chapter{Introduction}
% ----------------------------------------------------------
In the modern society, biometric technology is used in several security applications for personal authentication. The aim of such systems is to confirm a person identity based on physiological or behavioral traits. In the first case, recognition is based on biological characteristics such as fingerprint, palmprint, iris, face, etc. The latter relies on behavioral traits such as voice pattern and handwritten signature \cite{jain2004biometrics}.

Handwritten signature stands as one of the main approaches for identity authentication. One of the reasons for its widespread is the fact that the signature acquisition is easy and non-invasive, and most individuals are familiar with its use in their daily life \cite{impedovo2008state}. Due to its convenient nature, signatures can be employed as a sign of confirmation in a wide set of important documents, specifically, bank checks, credit card transactions, identification documents and a variety of business certificates and contracts.

As a behavioral trait, signatures are susceptible to spoof attacks, which is the attempt to forge the signature of a legitimate subject \cite{jain2004biometrics}. Two types of impostors are considered, specifically: casual impostors (producing random forgeries) when no information about authentic writer signature is known, and real impostors (producing skilled forgeries) when some information of the signature is used \cite{fierrez2008handbook}.

If a signature on a document is forged, this document is also considered invalid. Thus, prevent frauds in the signature verification process has been a challenge for researchers around the world. However, manual signature-based authentication of a large set of documents is difficult and also a time-consuming and labor-intensive task. Hence, several Automatic Handwritten Signature Verification Systems (AHSVS) have been proposed to support this task. These systems aim to automatically decide if a query signature is in fact of a particular person or not.

AHSVS are essentially a pattern recognition application that works by receiving a signature as input, extracting a feature set from the data and classifying the sample using a template database as reference. As any pattern recognition system, AHSVS are learning-based, which requires a dataset that can be used to assess their performances to create accurate signature verification methods. These datasets contain signatures digitized by either using an optical scanner to obtain the signature directly from the paper or by using an acquisition device such as digitizing tablets or electronic pens with digital ink.

The two approaches are identified as offline (static) and online (dynamic), respectively. In the online modality, data is stored during the writing process and consists of a temporal sequence of the two-dimensional coordinates (x, y) of consecutive points, whereas in the offline case, only a static representation of the completed writing is available as an image. Moreover, each representation has specific attributes not present in the other \cite{viard1999ireste}. For instance, online data do not include information about the width of the strokes and the texture of the ink on the paper. While the offline representation has lost all dynamic information of the writing process. As a result, features such as pen trajectory, which can be easily computed in the online domain, can only be inferred from a static image \cite{nel2005estimating}.


\section{Problem statement}
Encouraged by the motivations depicted previously, the goal of this dissertation can be stated as follows:
This work defines two approaches for testing software product line architectures defining activities, steps, inputs, outputs and roles in order to be confident that modifications
(correction or evolution) are conform with the architecture specification, do not introduce unexpected errors and that the new versions work as expected.

\section{Proposal outline}

\section{Statements of the contribution}

\section{Dissertation structure}




% However, as a behavioral trait, signatures are susceptible to spoof attacks, which is the attempt to spoof the signature of an enrolled user to fool the system \cite{jain2004biometrics}. Two types of impostors are considered, specifically: casual impostors (producing random forgeries) when no information about authentic writer signature is known, and real impostors (producing skilled forgeries) when some information of the signature is used \cite{fierrez2008handbook}.


%No entanto, já senti falta de uma discussão mais ampla referente aos problemas que nos motivaram a criar a nova base.
% **MOTIVAÇÃO**
In the last few years, several handwritten signature datasets have been created and some made publicly available. The general corpus consists of a set of genuine and forgery signatures for each writer and can be categorized on different dimensions including the modality (online or offline), script and size.

Although many datasets containing samples for offline, online or both modalities combined haven been proposed, those datasets normally do not convey some important real world challenges, not assessing the robustness of the systems on real world scenarios. Consequently, said systems often fail to deliver the expected results when employed in practice \cite{towards2013}.

In practical scenarios, signatures are acquired on a wide set of conditions and in both modalities. Different conditions for online acquisition includes signatures acquired in several types of devices, e.g., using smartphones or different models of digitizing tablets. Moreover, when dealing with offline signatures, most of the samples are present in documents with complex backgrounds and with different signing area constraints. Examples of such documents include bank checks, contracts, identification documents, forms, etc. \cite{towards2013, liwicki-sigseg}. Those distinct types of signatures often need to be integrated into the same system in an interoperable manner.

%There may be a case where the type of signature verification system used for training differs from that used for testing purpose. Though the test sample is of a genuine person, it might not be possible to prove with either of these systems alone. Hence, development of an integrated version of offline and online signature verification systems would be useful, either or both the offline and the online signature templates of the person being registered are recorded and an identification number is generated for that person. During testing, the test sample recorded is matched against the information available for that identification number in the database \cite{uppalapati2007integration}.

In regards to signature verification interoperability, many research problems are open to investigation, such as
\begin{inlinelist}
  \item development of complete document authentication systems involving both signature segmentation and verification process taking into account different signing area constraints
  \item analysis of the implications on AHSVS of the combination of signatures acquired on smartphones and conventional digitizing tablets
  \item development of systems capable of integrating both online and offline samples interchangeably, towards a unified signature biometry.
\end{inlinelist}
With the currently available datasets, investigation on the direction of the listed research problems is limited to samples acquired in controlled environments or can not be made at all.

Works have been done on topics directed towards signature verification interoperability. Qiao et al. \cite{qiao2007offline} proposed an offline signature verification system that uses online handwriting signatures instead of images in the registration phase, however, in the experiments the authors used synthetic offline images generated by the interpolation of online signature samples. Uppalapati \cite{uppalapati2007integration} proposed a system to integrate both modalities of handwritten signatures, not only providing a method to match offline signatures against an online and vice-versa, but also using both static and dynamic features, when available, to improve the system performance. Ahmed et al. \cite{liwicki-sigseg} proposed a method for signature extraction from documents, it is noteworthy that the segmentation accuracy was evaluated only on the patch level. According to the authors, it is due to the lack of publicly available datasets containing ground truth of signatures on the stroke level. Ahmed et al. in \cite{towards2013} discuss the currently non-applicability of most signature verification systems and the lack of complete document authentication systems involving signature segmentation and verification. According to the authors,  it is due to the absence of datasets suitable for the development of such systems, containing both patch and stroke level ground truth. Pirlo et al. \cite{pirlo2015interoperability} investigated the effects of signing area constraints on geometric features of online signatures. Diaz et al. \cite{diaz2014generation} proposed several approaches to synthetically generate offline signatures simulating the pen ink deposition on the paper based on dynamic information from online signatures. Zareen and Jabin \cite{zareen2016mobile} presented a comprehensive survey of mobile-biometric systems and proposed a method for online signature verification. The approach was evaluated on the SVC \cite{svc2004} dataset and a database acquired using a mobile device \cite{sgnote}.

Aiming to overcome the limitations of the current state of handwritten signature datasets, we present the RPPDI-SigData, an evaluation dataset for AHSVS that includes signatures captured for both online and offline modalities and from different signing conditions. Samples for the online modality were acquired on smartphones and digitizing tablets and for the offline domain acquired in documents with complex backgrounds (including a stroke level ground-truth) and different signing area constraints.

Alongside with the description of the RPPDI-SigData, this chapter also summarizes 17 publicly available handwritten signature datasets. Our goal is to provide the reader an overview of the existing evaluation datasets and its main characteristics such as number of samples, protocols, type of forgeries, script.

The rest of the chapter is structured as follows: Section 2 presents an overview of
the existing evaluation datasets for handwritten signature verification. Section 3 describes
RPPDI-SigData, our proposed dataset and we conclude the chapter in Section 4.

%!TEX root = ../dissertation_vkslm.tex

\chapter{Handwritten signature verification}

hello

%!TEX root = ../dissertation_vkslm.tex

\chapter{DEEP GENERATIVE MODELS}

\section{Neural Networks}
\section{Generative Adversarial Networks (GAN)}
\section{Autoencoder}
\section{Variational Autoencoder}
\subsection{DRAW}

%!TEX root = ../dissertation_vkslm.tex

\chapter{A model for offline signature generation from dynamic data}\label{ch:method}

%!TEX root = ../dissertation_vkslm.tex

\chapter{Experiments and results}

\section{Datasets}

\section{Experiments protocol}


%!TEX root = ../dissertation_vkslm.tex


\chapter{Conclusion}

\section{Contributions}

\section{Future work}

% ----------------------------------------------------------
% ELEMENTOS PÓS-TEXTUAIS
% ----------------------------------------------------------
\postextual
% ----------------------------------------------------------

% ----------------------------------------------------------
% Referências bibliográficas
% ----------------------------------------------------------
\bibliography{referencias}
% ----------------------------------------------------------
% Glossário
% ----------------------------------------------------------
%
% Consulte o manual da classe abntex2 para orientações sobre o glossário.
%
%\glossary

% ---
% Apêndices e Anexos
% ---
% \begin{apendicesenv}

%\chapter{Preprocessed IRONOFF samples}
%a
%\chapter{Neural Network model weights}
%a
%\chapter{Synthetic Samples from BiosecureID online signatures}
%a
%\chapter{Extracted Features}
%a
% ----------------------------------------------------------
\chapter{Publications}

As results of this work, a chapter has been published in the book Handwriting: Recognition, Development and Analysis and a paper has been submitted to the peer-reviewed journal Pattern Recognition Letters.

\textbf{Title:} Datasets for Handwritten Signature Verification: A Survey and a New Dataset, the RPPDI SigData.

\textbf{Authors:} Melo, V.K.S.L., Bezerra, B.L.D., do Nascimento, R.H.S.N., de Moura, G.C.D., Martins, G.L.L.d.S., Pirlo, G., Impedovo, D.

\textbf{In:} Handwriting: Recognition, Development and Analysis, ISBN 978-1-53611-957-2, Nova Science Publishers, New York, 2017

\textbf{Abstract:} As a result of a wide research in automatic handwritten signature verification, multiple data sets and competitions emerged to cover different tasks for online/offline signature verification and signature segmentation from scanned documents. The purpose of this work is to analyze and discuss the most used data sets in the literature in order to find what are the challenges pursued by the community in the past few years. The data sets, which were collected from recent publications and competitions, are described regarding its main characteristics such as the number of authors, signatures per author and the method used for the acquisition phase. We analyzed several databases used for different topics related to either online or offline signature verification. Since we did not find in this review a data set that can be used to explore signature verification considering different acquisition sources and writing conditions, specifically a data set consisting of dynamic signatures taken from multiple devices and static signatures with diverse signing area constraints and background textures, we introduce a new data set acquired from pen tablets, smartphones and scanned from different types of documents. Therefore, researchers can use this new data set not only to investigate the multi-domain signature verification problem but to take into account in their solutions the signature segmentation problem in real world documents with complex backgrounds.

\hspace{0.6in}

\textbf{Title:} A Deep Learning Approach to Generate Off-line Handwritten Signatures Based on On-line Samples

\textbf{Authors:} Melo, V.K.S.L., Bezerra, B.L.D., Impedovo, D., Pirlo, G.

\textbf{(In revision):} Pattern Recognition Letters

\textbf{Abstract:} One of the main challenges of offline signature verification is the absence of large databases. A possible alternative to overcome this problem is the generation of synthetic signature databases. In this work, a novel method for the generation of synthetic off-line signatures based on dynamic information is presented. In contrast to the state-of-the-art, we propose a synthesis approach under the perspective of supervised training, in which our learning model is trained to perform the task of “online signature to offline signature conversion”. The proposed approach is based on a Deep Convolutional Neural Network trained to learn how online handwritten manuscripts of the IRONOFF dataset are transformed into the offline domain. The main goal of the proposed method is to synthetically enlarge existing offline signature datasets based on online signature samples towards an improvement on the recognition rates of offline signature verification systems. For these purposes, a machine-oriented evaluation on the BiosecurID signature dataset is carried out. Through the synthetic samples generated with the proposed method, we show a verification performance similar to the one offered by real authentic signatures and assuring improvements of the Equal Error Rate results in comparison with the current state-of-the-art method.


\end{apendicesenv}

%\begin{anexosenv}


% ---
\chapter{Morbi ultrices rutrum lorem.}
% ---
\lipsum[30]

% ---
\chapter{Cras non urna sed feugiat cum sociis natoque penatibus et magnis dis
parturient montes nascetur ridiculus mus}
% ---

\lipsum[31]

% ---
\chapter{Fusce facilisis lacinia dui}
% ---

\lipsum[32]

\end{anexosenv}



%---------------------------------------------------------------------
% INDICE REMISSIVO
%---------------------------------------------------------------------
\phantompart
\printindex
%---------------------------------------------------------------------

\end{document}

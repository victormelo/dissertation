 \begin{apendicesenv}

%\chapter{Preprocessed IRONOFF samples}
%a
%\chapter{Neural Network model weights}
%a
%\chapter{Synthetic Samples from BiosecureID online signatures}
%a
%\chapter{Extracted Features}
%a
% ----------------------------------------------------------
\chapter{Publications}

As results of this work, a chapter has been published in the book Handwriting: Recognition, Development and Analysis and a paper has been submitted to the peer-reviewed journal Pattern Recognition Letters.

\textbf{Title:} Datasets for Handwritten Signature Verification: A Survey and a New Dataset, the RPPDI SigData.

\textbf{Authors:} Melo, V.K.S.L., Bezerra, B.L.D., do Nascimento, R.H.S.N., de Moura, G.C.D., Martins, G.L.L.d.S., Pirlo, G., Impedovo, D.

\textbf{In:} Handwriting: Recognition, Development and Analysis, ISBN 978-1-53611-957-2, Nova Science Publishers, New York, 2017

\textbf{Abstract:} As a result of a wide research in automatic handwritten signature verification, multiple data sets and competitions emerged to cover different tasks for online/offline signature verification and signature segmentation from scanned documents. The purpose of this work is to analyze and discuss the most used data sets in the literature in order to find what are the challenges pursued by the community in the past few years. The data sets, which were collected from recent publications and competitions, are described regarding its main characteristics such as the number of authors, signatures per author and the method used for the acquisition phase. We analyzed several databases used for different topics related to either online or offline signature verification. Since we did not find in this review a data set that can be used to explore signature verification considering different acquisition sources and writing conditions, specifically a data set consisting of dynamic signatures taken from multiple devices and static signatures with diverse signing area constraints and background textures, we introduce a new data set acquired from pen tablets, smartphones and scanned from different types of documents. Therefore, researchers can use this new data set not only to investigate the multi-domain signature verification problem but to take into account in their solutions the signature segmentation problem in real world documents with complex backgrounds.

\hspace{0.6in}

\textbf{Title:} A Deep Learning Approach to Generate Off-line Handwritten Signatures Based on On-line Samples

\textbf{Authors:} Melo, V.K.S.L., Bezerra, B.L.D., Impedovo, D., Pirlo, G.

\textbf{(In revision):} Pattern Recognition Letters

\textbf{Abstract:} One of the main challenges of offline signature verification is the absence of large databases. A possible alternative to overcome this problem is the generation of synthetic signature databases. In this work, a novel method for the generation of synthetic off-line signatures based on dynamic information is presented. In contrast to the state-of-the-art, we propose a synthesis approach under the perspective of supervised training, in which our learning model is trained to perform the task of “online signature to offline signature conversion”. The proposed approach is based on a Deep Convolutional Neural Network trained to learn how online handwritten manuscripts of the IRONOFF dataset are transformed into the offline domain. The main goal of the proposed method is to synthetically enlarge existing offline signature datasets based on online signature samples towards an improvement on the recognition rates of offline signature verification systems. For these purposes, a machine-oriented evaluation on the BiosecurID signature dataset is carried out. Through the synthetic samples generated with the proposed method, we show a verification performance similar to the one offered by real authentic signatures and assuring improvements of the Equal Error Rate results in comparison with the current state-of-the-art method.


\end{apendicesenv}

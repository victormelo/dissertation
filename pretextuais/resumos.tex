% resumo em inglês
\setlength{\absparsep}{18pt} % ajusta o espaçamento dos parágrafos do resumo
\begin{resumo}

One of the main challenges of off-line signature
verification is the absence of large databases. A possible
alternative to overcome this problem is the generation of synthetic signature databases. In this work, a novel method for the generation of synthetic off-line signatures based on dynamic
information is presented. In contrast to the state-of-the-art, we propose an synthesis approach under the perspective of supervised training, in which our learning model is trained to perform the task of ``on-line signature to off-line signature conversion''. The proposed approach is based on a Deep Convolutional Neural Network trained to learn how on-line handwritten manuscripts of the IRONOFF dataset are transformed into the off-line domain. The main goal of the proposed method is to synthetically enlarge existing off-line signature datasets based on on-line signature samples towards an improvement on the recognition rates of off-line signature verification systems. For these purposes, a machine-oriented evaluation on the BiosecurID signature dataset is carried out. We show that the synthetic samples generated with our proposed method achieve a verification performance similar to the one offered by real signatures and promising improvements of the Equal Error Rate results in comparison with and the current state-of-the-art method.


\end{resumo}

% resumo em português
\begin{resumo}[Resumo]
 \begin{otherlanguage*}{brazil}
  Um dos principais desafios de sistemas de verificação de assinaturas \textit{off-line} é a ausência de grandes conjunto de dados. Uma alternativa possível para superar esse probema é a geração de assinaturas sintéticas. Neste trabalho é proposto um método para a geração sintética de assinaturas off-line baseado em informações dinâmicas. Em contraste com o estado-da-arte, o método método de síntese proposto se baseia na perspectiva da aprendizagem supervisionada, o nossa máquina de aprendizagem é treinada para realizar a tarefa de ``conversão de assinatura \textit{on-line} para \textit{off-line}''. O método proposto é uma \textit{Deep Convolutional Neural Network} treinada para aprender como textos manuscritos \textit{on-line} da base IRONOFF são transformados para o domínio \textit{off-line}. O objetivo principal do método proposto é o de aumentar sinteticamente bases de assinatura \textit{off-line} baseando-se em amostras \textit{on-line} em direção a uma melhora nas taxas de reconhecimento de sistemas de verificação de assinaturas \textit{off-line}. Para isso, uma avaliação na base de assinaturas BiosecurID é realizada. Mostra-se que as amostras sintéticas geradas pelo método proposto obtém uma performance de verificação similar aos oferecidos por assinaturas reais e uma melhora promissora no \textit{Equal Error Rate} em comparação com o método do estado-da-arte.

   \vspace{\onelineskip}
 
   \noindent 
 \end{otherlanguage*}
\end{resumo}


% ---

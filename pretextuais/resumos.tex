% resumo em inglês
\setlength{\absparsep}{18pt} % ajusta o espaçamento dos parágrafos do resumo
\begin{resumo}
One of the main challenges of off-line signature verification is the absence of large databases. A possible alternative to overcome this problem is the generation of synthetic signature databases. In this work, a novel method for the generation of synthetic off-line signatures based on dynamic information is presented. In contrast to the state-of-the-art, we propose a synthesis approach under the perspective of supervised training, in which our learning model is trained to perform the task of ``online signature to off-line signature conversion''. The proposed approach is based on a Deep Convolutional Neural Network trained to learn how online manuscripts of the IRONOFF dataset are transformed into the offline domain. The main goal of the proposed method is to synthetically enlarge existing offline signature datasets when complementary online signatures are avaiable towards an improvement on the recognition rates of off-line signature verification systems. For these purposes, a machine-oriented evaluation on the BiosecurID signature dataset is carried out. We show that our proposed method model can learn how the online biometric data must be used to build the grayscale signature image. Specifically, we observe that the synthetic samples generated with our proposed method are comparable to real signatures, achieving a verification performance similar to the one offered by real signatures. We also show that the proposed system can be combined with any other offline signature verification system to increase the number of samples on the enrollment set, leading to improved Equal Error Rate, including when used to detect skilled forgeries.
\end{resumo}

% resumo em português
\begin{resumo}[Resumo]
 \begin{otherlanguage*}{brazil}
  Um dos principais desafios de sistemas de verificação de assinaturas \textit{offline} é a ausência de grandes conjunto de dados. Uma alternativa possível para superar esse probema é a geração de assinaturas sintéticas. Neste trabalho é proposto um método para a geração sintética de assinaturas offline baseado em características dinâmicas. Em contraste com o estado-da-arte, o método  proposto se baseia na perspectiva da aprendizagem supervisionada, o nossa máquina de aprendizagem é treinada para realizar a tarefa de ``conversão de assinatura \textit{online} para \textit{offline}''. O método proposto é uma \textit{Deep Convolutional Neural Network} treinada para aprender como textos manuscritos \textit{online} da base IRONOFF são transformados para o domínio \textit{offline}. O objetivo principal do método proposto é o de aumentar sinteticamente bases de assinatura \textit{offline} baseando-se em amostras \textit{online} em direção a uma melhora nas taxas de reconhecimento de sistemas de verificação de assinaturas \textit{offline}. Para isso, uma avaliação na base de assinaturas BiosecurID é realizada. Mostra-se que o modelo de rede profunda proposto consegue aprender como converter os dados biométricos \textit{online} para criar a respectiva imagem em escala de cinza da assinatura. Especificamente, observa-se que as amostras sintéticas geradas com o método proposto são comparáveis a amostras reais, obtendo uma performance comparável ao de assinaturas reais. Também é mostrado que o sistema proposto pode ser combinado com qualquer sistema de verificação de assinatura \textit{offline} para aumentar o número de amostras no conjunto de treinamento, levando a uma melhora no \textit{EER}, inclusive quando usado para detectar forjas habilidosas.
  
   \vspace{\onelineskip}
 
   \noindent 
 \end{otherlanguage*}
\end{resumo}


% ---

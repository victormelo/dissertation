\documentclass{beamer}
\mode<presentation>
\usepackage{beamerthemesplit}

\usepackage{verbatim}
\usepackage{hyperref}

%\usepackage{pgfpages}
%\pgfpagesuselayout{4 on 1}[landscape,letterpaper,border shrink=2.5mm]





\usetheme{Berlin}
\usecolortheme{albatross}
\setbeamerfont*{frametitle}{size=\normalsize}
\setbeamertemplate{navigation symbols}{



}

% Setup TikZ

\usepackage{tikz}


\usetikzlibrary{positioning}
% The face style, can be changed
\tikzset{face/.style={shape=circle,minimum size=4ex,shading=radial,outer sep=0pt,
        inner color=white!50!yellow,outer color= yellow!70!orange}}
%% Some commands to make the code easier
\newcommand{\emoticon}[1][]{%
  \node[face,#1] (emoticon) {};
  %% The eyes are fixed.
  \draw[fill=white] (-1ex,0ex) ..controls (-0.5ex,0.2ex)and(0.5ex,0.2ex)..
        (1ex,0.0ex) ..controls ( 1.5ex,1.5ex)and( 0.2ex,1.7ex)..
        (0ex,0.4ex) ..controls (-0.2ex,1.7ex)and(-1.5ex,1.5ex)..
        (-1ex,0ex)--cycle;}
\newcommand{\pupils}{
  %% standard pupils
  \fill[shift={(0.5ex,0.5ex)},rotate=80]
       (0,0) ellipse (0.3ex and 0.15ex);
  \fill[shift={(-0.5ex,0.5ex)},rotate=100]
       (0,0) ellipse (0.3ex and 0.15ex);}

\newcommand{\emoticonname}[1]{
  \node[below=1ex of emoticon,font=\footnotesize,
        minimum width=4cm]{#1};}

\usetikzlibrary{trees}

\title[LaTeX:Presentations]{Latex: Presentations Using Beamer and Tikz}
\subtitle[]{Biometry 789-02}
\author{Emily Kistner-Griffin}
\date{April 12, 2012}
\institute[MUSC]{}
\pgfdeclareimage[height=.7cm]{university-logo}{MUSCimage}
\logo{\pgfuseimage{university-logo}}
\begin{document}
\begin{frame}
\titlepage
\end{frame}

{
  \begin{frame}<beamer>{Outline}
    \tableofcontents[pausesections]
  \end{frame}
}

\section{Introduction to Beamer}
\subsection{}
\begin{frame}[fragile]
  \frametitle{What is Beamer?}
  \begin{itemize}
  \item Beamer is a LaTeX document class for producing slides created by Til Tantau at the University of Leubeck
  \item Original version from 2003
  \item Makes creating PDF presentations with bells and whistles straightforward
  \item A guide to help you get started can be found: \url{http://www.math.umbc.edu/~rouben/beamer/quickstart.html}
  \item You have learned \verb|\documentclass{article}|
  \item Today we are discussing \verb|\documentclass{beamer}|
   \end{itemize}
\end{frame}

\frame {
  \frametitle{Beamer: Advantages}
  \begin{itemize}
  \item Unlike PowerPoint (particularly when using Equation Editor or importing figures), presentation will appear the same regardless of computer (MAC, PC)
  \item Once you learn basic LaTeX commands, you can create presentations with varying layouts etc
  \item Creates an automatic table of contents with clickable links (see header)
  \item Themes allow changing appearance of the presentation
  \item Inclusion of overlays and dynamic effects
  \end{itemize}
}

\begin{frame}[fragile]
  \frametitle{Beamer: Advantages}
  \begin{itemize}
  \item If you are writing your dissertation in LaTeX it is easy to cut and paste code to make conference/defense presentations or vice versa
  \item Useful templates available with the Beamer download:
  \begin{verbatim}C:\Program Files\MikTex 2.9\doc\latex\beamer
  \solutions\ \end{verbatim}
  \item A 247 page user guide is also available in the same folder:
  \begin{verbatim}C:\Program Files\MikTex 2.9\doc\latex\beamer\
  doc\beameruserguide.pdf\end{verbatim} - for the serious presenter!
  \end{itemize}
\end{frame}

\frame {
  \frametitle{Beamer: Templates}
  \begin{itemize}
  \item Let's try a template
  \item Open template from class website and copy into WinEdt
  \item Select tab "Tex" and PDF and PDFtexify
  \item Nice elements not available in PowerPoint - table of contents, links to sections and subsections, etc
  \end{itemize}
}

\frame {
  \frametitle{Beamer: Themes}
  \begin{itemize}
  \item Beamer document class allows the user to select one of many themes to specify appearance
  \item This lecture uses the theme Darmstadt
  \item Many other themes are available: default, Boadilla, Madrid, Pittsburgh, Rochester, Copenhagen, Warsaw, Singapore, Malmoe, etc
   \end{itemize}
}

\begin{frame}[fragile]
  \frametitle{Colors}
 To change the colors of the presentation you need to change
 \verb|\usecolortheme{default}| in the preamble (before you begin the document)
 \begin{block}{Color Options}
 \begin{center}
 albatross crane beetle dove fly seagull wolverine beaver
 \end{center}
\end{block}
\end{frame}

\section{The Basics}
\subsection{}
\begin{frame}[fragile]
  \frametitle{Title Page}
  \begin{itemize}
  \item Very easy to change \verb|\title|, \verb|\subtitle|, \verb|\author|, \verb|\institute|, \verb|\date in template|
  \item Notice [short paper title] for shorter titles, dates etc that display throughout presentation
  \item Notice \verb|%| for commenting code
  \end{itemize}
\end{frame}

\begin{frame}[plain,fragile]
  \frametitle{Frames}
  \begin{itemize}
  \item Each slide is coded as a frame: \verb|\begin{frame}| and \verb|end{frame}|
  \item Can also code as \verb|\frame{ ... }|
  \item Notice how \verb|\titlepage| and \verb|\tableofcontents| are specified
  \item Sometimes I use the \verb|[pausesections]| option after \verb|\tableofcontents|
  \item Specify titles on each slide with \verb|\frametitle{}| or with \verb|\begin{frame}{TITLE}|
  \item Notice I can get rid of headers and footers with the frame option \verb|[plain]|
  \end{itemize}
\end{frame}

\begin{frame}[fragile]
  \frametitle{Sections and Subsections}
  \begin{itemize}
  \item To create a section: \verb|\section{TITLE}|
  \item Notice my sections in the header: Introduction to Beamer, Getting Started, etc
  \item Also determines entries in the table of contents
  \item Section and subsection commands are given \textit{outside} of frames!
  \end{itemize}
\end{frame}

\begin{frame}[fragile]
  \frametitle{Environments: Definitions}
  \begin{itemize}
  \item If you want to define something, specify \verb|\begin{definition}| and \verb|\end{definition}|:
  \begin{definition}
  $\pi$ is a mathematical constant that is the ratio of a circle's circumference to it's diameter.
  \end{definition}
  \end{itemize}
\end{frame}

\begin{frame}[fragile]
  \frametitle{Environments: Theorems, Lemmas, Proofs, Corollaries, Examples}
  \begin{itemize}
  \item If you want to highlight an example, specify \verb|\begin{example}| and \verb|\end{example}|:
  \begin{example}{PDF for the Cauchy Distribution}
  \[f(x)= \frac{1}{\pi(1+x^{2})}\]
  \end{example}
  \end{itemize}
\end{frame}

\begin{frame}[fragile]
  \frametitle{Generic Environments: Block}
  \begin{itemize}
  \item If you want to highlight any text, specify \verb|\begin{block}{TITLE}| and \verb|\end{block}|:
  \begin{block}{PDF for the Cauchy Distribution}
  \[f(x)= \frac{1}{\pi(1+x^{2})}\]
  \end{block}
  \end{itemize}
\end{frame}

\begin{frame}[fragile]
  \frametitle{Frame Layout}
  \begin{columns}
  \column{.5\textwidth}
  Column 1 can be specified with \verb|\begin{columns}| and \verb|\column{.5\textwidth}|
  \column{.5\textwidth}
  Column 2 specified by with \verb|\column{.5\textwidth}| and \verb|\end{columns}|
  \end{columns}
\end{frame}


\begin{frame}[fragile]
  \frametitle{Overlays}
  \begin{itemize}
  \item I use bullets on most slides with: \verb|\begin{itemize}| and \verb|\end{itemize}|
  \pause
  \item Each bullet is specified with: \verb|\item|
  \pause
  \item Then add \verb|\pause| after each item
  \pause
  \item This allows showing each bullet incrementally
  \end{itemize}
\end{frame}

\begin{frame}[fragile]
  \frametitle{Overlays}
  \begin{enumerate}
  \item You can also enumerate with: \verb|\begin{enumerate}| and \verb|\end{enumerate}|
  \pause
  \item Each number is specified with: \verb|\item|
  \pause
  \item Then add \verb|\pause| after each item
  \pause
  \item This allows showing each number incrementally
  \end{enumerate}
\end{frame}

\section{Adding Bells and Whisles}
\subsection{}
\begin{frame}[fragile]
  \frametitle{Getting Complicated: Overlays}
  \begin{itemize}
  \item<1-> Show only the 1st item with \verb|\item<1->|
  \item<2-> Then add each additional item by specifying \verb|\item<2->|
  \item<3-> Then add each item with with \verb|\item<3->|
  \item<4-> You don't need \verb|\pause| just specify the order in the \verb|<>|
  \item<1-> and last item with \verb|\item<1->|!
  \end{itemize}
\end{frame}

\begin{frame}[fragile]
  \frametitle{Getting Complicated: Overlays without Itemizing}
  \uncover<1->{Without bullets use \textit{uncover}}
  \uncover<2->{instead of \textit{item}.}
  \uncover<3->{Or you can use \textit{only} instead of \textit{item}.}
  \only<4->{You still need to specify on which slide the text should appear.}
  \only<5->{But this is less readable.}
\end{frame}

\begin{frame}[fragile]
  \frametitle{Getting Complicated: More Overlays}
  \begin{itemize}
  \item Highlight only on some slides
  \item \alert<2>{For example only highlight this slide 2}
  \item To do this use \verb|\alert<2>{}|
  \end{itemize}
\end{frame}

\begin{frame}[fragile]
  \frametitle{Getting Complicated: More Overlays}
  \begin{itemize}
  \item Instead of highlighting you can change the colors of items
  \item \color<2>{green}For example make green and blue bullets
  \item \color<2>{blue}To do this use \verb|\color<2>{green}| or \verb|\color<2>{blue}|
  \end{itemize}
\end{frame}

\begin{frame}[fragile]
  \frametitle{Graphics}
 Let's add an image with code:
 \verb|\includegraphics[height]{filename}|
 \begin{center}
  \includegraphics[height=35mm]{DNA.png}
  \end{center}
\end{frame}

\begin{frame}[fragile]
  \frametitle{Hyperlinks}
 Let's add an hyperlink with code:
 \verb|\url{http://people.musc.edu/~elg26/teaching/...}|
 \begin{center}
  \color{blue}
  \url{http://people.musc.edu/~elg26/teaching/statcomputing.2012/statcomputingI.2012.htm}
  \end{center}
\end{frame}

\begin{frame}[fragile]
  \frametitle{Handouts}
 \begin{itemize}
 \item Sometimes it's nice to provide handouts for your presentation
 \item Need to change the document class option to \verb|documentclass[handout]{beamer}|
 \item You also probably want to save paper by printing multiple slides/page
     \begin{center}
     \begin{verbatim}
 \usepackage{pgfpages}
 \pgfpagesuselayout{4 on 1}
 [landscape,letterpaper,border shrink=2.5mm]
     \end{verbatim}
     \end{center}
 \end{itemize}
\end{frame}

\section{Adding Tikz}
\subsection{}
\begin{frame}[fragile]
  \frametitle{Drawing in Beamer}
 \begin{itemize}
 \item You can even draw in Beamer!
 \item Need to add the tikz package \verb|\usepackage{tikz}|
 \item To start drawing \verb|\begin{tikzpicture}| and you know how to end the picture
 \item You need to end tikz commands with ;
 \end{itemize}
 \begin{center}
 \begin{tikzpicture}
 \emoticon[inner color=white!50!red,outer color= red!70!red!90!black]
    \pupils\emoticonname{devilish}
    %% mouth
    \draw[thick,line cap=round] (-1ex,-1ex)
         ..controls (-0.5ex,-1.5ex)and(0.5ex,-1.5ex)..(1ex,-1ex);
    %% tail
    \draw[very thick,-stealth,red!90!black] (emoticon.330)--++(330:0.01ex)
         ..controls (3ex,-3ex)and(3.5ex,1ex)..(4ex,-2ex);
    %% horns
    \draw ( emoticon.80)..controls ( 0.6ex,2.4ex)..( 1ex,2.5ex)
          ..controls ( 0.8ex,2.3ex)..(emoticon.70);
    \draw (emoticon.100)..controls (-0.6ex,2.4ex)..(-1ex,2.5ex)
          ..controls (-0.8ex,2.3ex)..(emoticon.110);

 \end{tikzpicture}
 \end{center}
\end{frame}

\begin{frame}
\frametitle{More Realistically}
\begin{center}
 \begin{tikzpicture}

\def \n {5}
\def \radius {3cm}
\def \margin {8} % margin in angles, depends on the radius

\foreach \s in {1,...,\n}
{
  \node[draw, circle] at ({360/\n * (\s - 1)}:\radius) {$\s$};
  \draw[->, >=latex] ({360/\n * (\s - 1)+\margin}:\radius)
    arc ({360/\n * (\s - 1)+\margin}:{360/\n * (\s)-\margin}:\radius);
}
\end{tikzpicture}
\end{center}
\end{frame}



\section{Discussion}

\subsection{}
\frame {
   \frametitle{}
   \begin{center}
   Questions?
   \end{center}
 }

\end{document}











